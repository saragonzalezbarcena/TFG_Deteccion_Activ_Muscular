\capitulo{1}{Objetivos}

%Objetivos principales del trabajo realizado.

%Este apartado explica de forma precisa y concisa cuales son los objetivos que se persiguen con la realización del proyecto. Se puede distinguir entre:
%\begin{enumerate}
  %  \item Los objetivos marcados por los requisitos del software/hardware/análisis a desarrollar.
   % \item Los objetivos de carácter técnico, relativos a la calidad de los resultados, velocidad de ejecución, fiabilidad o similares.
    %\item Los objetivos de aprendizaje, relativos a aprender técnicas o herramientas de interés. 
%\end{enumerate}

En las asociaciones para el cuidado y la mejora de la calidad de vida de las personas con Párkinson se requieren dispositivos especiales que puedan analizar la actividad de los pacientes como la duración del ejercicio, el número de bloqueos del paciente durante el período de ejercicio y, si hay un desequilibrio entre ambos lados del cuerpo o si un lado está poniendo más fuerza que el otro.

Sin embargo, los dispositivos que son capaces de capturar este tipo de datos son muy específicos y poco comunes, por lo que son costosos.

El objetivo es crear un dispositivo sensorial que pueda capturar todas las actividades musculares, analizarlas y almacenarlas con el fin de que los especialistas puedan utilizarlas posteriormente para registrar los ejercicios de los pacientes y emplear estos datos en la mejora de la calidad de vida de estos. Asimismo, el proyecto está destinado a enviar los datos en tiempo real a través de internet/intranet para la observación continua de los pacientes.

Constará de una primera parte destinada a capturar y analizar las señales de todos los músculos necesarios y, de una segunda orientada a recibirlas y almacenarlas para una buena visualización de los datos. Por último, se enviarán a través de intranet/internet para la observación continua del paciente. 









