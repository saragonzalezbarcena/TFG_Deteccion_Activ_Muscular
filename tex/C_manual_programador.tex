\apendice{Manual del desarrollador / programador / investigador.} % usar el término que mejor se corresponda.

\section{Estructura de directorios}

Descripción de los directorios y ficheros entregados.

Todas las carpetas y ficheros se encuentran en el repositorio.

Para realizar la lectura del sensor MPU6050, se emplearán dos biblioteca de Arduino desarrolladas por Jeff Rowberg \cite{website:github.com/jrowberg}, las cuales se encuentran en la carpeta \textit{libraries}.

\begin{itemize}
    \item \textit{libraries/MPU6050}: biblioteca para MPU6050.
    \item \textit{libraries/I2Cdev}: biblioteca para la comunicación I2C.
\end{itemize}

La carpeta \textit{Arduino} contiene los archivos correspondientes con el código necesario para el funcionamiento de los sensores.

\begin{itemize}
    \item \textit{Arduino/MPU6050/MPU6050.ino}: inluye el código para calcular los ángulos de inclinación y de rotación del acelerómetro y gioscopio respectivamente.

\end{itemize}

\section{Compilación, instalación y ejecución del proyecto}

En caso de ser necesaria esta sección, porque la compilación o ejecución no sea directa.


\section{Pruebas del sistema}
Esta sección puede ser opcional.

Puede tratarse de validación de la interfaz por parte de los usuarios, mediante escuestas o similar o validación del funcionamiento mediante pruebas unitarias.



\section{Instrucciones para la modificación o mejora del proyecto.}

Instrucciones y consejos para que el trabajo pueda ser mejorado en futuras ediciones.